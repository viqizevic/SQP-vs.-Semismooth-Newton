\chapter{Einf�hrung}

Die nichtlineare Optimierung ist ein bedeutendes Gebiet der Mathematik.
Sie findet immer wieder Anwendungen in den schwierigen Problemen der Technik und
der Wirtschaft. Es wurden viele Verfahren entwickelt, um nichtlineare
Optimierungsprobleme zu l�sen. In dieser Arbeit werden zwei Verfahren, das
halbglatte Newton-Verfahren und das SQP-Verfahren, betrachtet und verglichen.

Das SQP-Verfahren geh�rt zu den bekanntesten Verfahren der nichtlinearen
Optimierung. Es wurde schon seit den 60er Jahren entwickelt und wurde in vielen
Optimierungsproblemen als Standardwerkzeug angewendet sowie weiterentwickelt.
Das halbglatte Newton-Verfahren ist weniger bekannt als das SQP-Verfahren. Es basiert aber auf
das bekannte Newton-Verfahren.

``In his 1963 PhD thesis, Wilson proposed the first sequential quadratic
programming (SQP) method for the solution of constrained nonlinear optimization problems. In the
intervening 48 years, SQP methods have evolved into a powerful and effective
class of methods for a wide range of optimization problems.''
(Gill \& Wong 2010 \cite{GiWo})

``It was not predictable in 2000 that ten years later semismooth Newton
methods would be one of the most important approaches for solving inequality
constrained optimization problems in function spaces.''
(Ulbrich 2011 \cite{Ulb})