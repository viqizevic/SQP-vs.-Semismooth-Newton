\addchap{Einleitung}

Die nichtlineare Optimierung ist ein bedeutendes Gebiet der Mathematik.
Sie findet immer wieder Anwendungen in schwierigen Problemen
der Technik und der Wirtschaft.
Viele Verfahren wurden entwickelt,
um nichtlineare Optimierungsprobleme zu l�sen.
In dieser Arbeit werden zwei Verfahren verglichen,
n�mlich das halbglatte Newton-Verfahren und das SQP-Verfahren.
Es liegt nahe, zu untersuchen,
welches von den beiden Verfahren besser ist.

Das SQP-Verfahren geh�rt zu den bekanntesten Verfahren
der nichtlinearen Optimierung.
In seiner Doktorarbeit hat Wilson 1963
das erste SQP-Verfahren entwickelt~\cite{wilson}.
Es wird bis heute in vielen Optimierungsproblemen als
Standardwerkzeug angewendet und wurde inzwischen sehr viel
weiterentwickelt.

Das halbglatte Newton-Verfahren ist nicht so bekannt wie das SQP-Verfahren.
Es basiert aber auf dem bekannten Newton-Verfahren.
�ber dieses Verfahren hat Ulbrich in \cite{ulbrich} geschrieben:
\textss{It was not predictable in 2000 that ten years later semismooth Newton
methods would be one of the most important approaches for solving inequality
constrained optimization problems in function spaces.}
% Das war nicht vorhersehbar im Jahr 2000, dass 10 Jahre sp�ter das halbglatte
% Newton-Verfahren einer der wichtigsten Ans�tze geworden ist, um
% Optimierungsprobleme mit Ungleichungsnebenbedingungen
% in Funktionenr�umen zu l�sen

Die Gliederung der Arbeit sieht wie folgt aus.
Zuerst werden die wichtigsten
Grundlagen der Optimierungstheorie eingef�hrt.
In Kapitel~\ref{chap:seq_quad_prog} wird das SQP-Verfahren betrachtet
und im darauf folgenden Kapitel das halbglatte Newton-Verfahren.
Anschlie�end werden in Kapitel~\ref{chap:vergleich} die numerische
Ergebnisse der beiden Verfahren f�r verschiedene Testprobleme
aufgef�hrt und verglichen.