\chapter{Einf�hrung}

Die nichtlineare Optimierung ist ein bedeutendes Gebiet der Mathematik.
Sie findet immer wieder Anwendungen in den schwierigen Problemen der Technik und
der Wirtschaft. Es wurden viele Verfahren entwickelt, um nichtlineare
Optimierungsprobleme zu l�sen. In dieser Arbeit werden zwei Verfahren, das
halbglatte Newton-Verfahren und das SQP-Verfahren, betrachtet und verglichen.

Das SQP-Verfahren geh�rt zu den bekanntesten Verfahren der nichtlinearen
Optimierung. Es wurde schon seit den 60er Jahren entwickelt und wurde in vielen
Optimierungsproblemen als Standardwerkzeug angewendet sowie weiterentwickelt.
Das halbglatte Newton-Verfahren ist weniger bekannt als das SQP-Verfahren. Es basiert aber auf
das bekannte Newton-Verfahren.

% TODO tell more about SQP and Semismooth-Newton

\section{Allgemeine Optimierungsprobleme}

\begin{definition}
Allgemein ist die Aufgabenstellung der nichtlinearen Optimierung wie folgt
definiert:
\begin{equation}
  \min_{x \in \F} f(x) \label{P} %TODO set to (P) not (1.1)
\end{equation}
Die Funktion $f:D \subseteq \R^n \rightarrow \R$ ist die sogennante Zielfunktion.
$D$ sei der Definitionsbereich von $f$.
$\F$ sei eine nichtleere Teilmenge von $D$, die man als L�sungsmenge bezeichnet.
Alle Elemente von $\F$ werden als zul�ssige Punkte bezeichnet.
$\F$ wird durch die sogennanten Nebenbedingungen definiert.
\end{definition}

Man kann hierbei den Unterschied zwischen der linearen Optimierung und
der nichtlinearen Optimierung erkennen.
Bei der linearen Optimierung muss die Zielfunktion linear sein
(d.h. die Zielfunktion besitzt die Form $f(x) = c^T x$, $c \in \R^n$)
und die Nebenbedingungen sind durch lineare Gleichungssysteme oder
Ungleichungssyteme definiert.
Bei der nichtlinearen Optimierung gibt es dagegen keine Einschr�nkung,
wie die Zielfunktion und die Nebenbedingungen aussehen sollen.
F�r die lineare Optimierung ist ein in der Praxis sehr effizientes Verfahren
bekannt. Aber die Verfahren der nichtlinearen Optimierung sind auch f�r die
linearen Optimierungsprobleme anwendbar. % TODO: Stimmt das?
Umgekehrt ist es jedoch nicht m�glich.

Ein einfaches Beispiel nichtlinearer Optimierung ist das Problem
\[
  \min_{x \in \R} (x-1)^2.
\]

Falls $\F = D$ gilt,
dann bezeichnet man das Optimierungsproblem als unrestringiert.
Es besitzt also keine Nebenbedingungen.
Ansonsten hei�t es ein restingiertes Optimierungsproblem.

Es ist �blich, dass man nur Minimierungsproblem betrachtet, weil ein
Maximierungsproblem $\max g(x)$ zu dem Minimierungsproblem $\min f(x) := -g(x)$
�quivalent ist.

\begin{definition}
\emph{(Globale und lokale L�sung)}\\
Ein Punkt $\xopt \in \F$ hei�t globale L�sung des Problems~(\ref{P}) oder
globales Minimum von $f$, wenn
\begin{equation}
  f(\xopt) \leq f(x) \qquad \forall x \in \F
\end{equation}
gilt.
Ein Punkt $\xopt \in \F$ hei�t strikte globale L�sung des Problems~(\ref{P}) oder
striktes globales Minimum von $f$, wenn
\begin{equation}
  f(\xopt) < f(x) \qquad \forall x \in \F\backslash\{\xopt\}
\end{equation}
gilt.
Ein Punkt $\xopt \in \F$ hei�t lokale L�sung des Problems~(\ref{P}) oder
lokales Minimum von $f$, wenn f�r eine Umgebung $U(\xopt)$ von $\xopt$
\begin{equation}
  f(\xopt) \leq f(x) \qquad \forall x \in U(\xopt) \cap \F
\end{equation}
gilt.
Ein Punkt $\xopt \in \F$ hei�t strikte lokale L�sung des Problems~(\ref{P}) oder
striktes lokales Minimum von $f$, wenn f�r eine Umgebung $U(\xopt)$ von $\xopt$
\begin{equation}
  f(\xopt) < f(x) \qquad \forall x \in U(\xopt) \cap \F\backslash\{\xopt\}
\end{equation}
gilt.
Eine Umgebung von $\xopt$ ist einfach eine offene Menge, die $\xopt$ beinhaltet.
\end{definition}

Wegen dieser Definitionen kommt die Unterscheidung zwischen der globalen
und der lokalen Optimierung. Globale Optimierung versucht, globale L�sungen
zu finden. Lokale Optimierung versucht dagegen, lokale L�sungen zu finden.
Viele Verfahren finden lokale L�sungen, die jedoch nicht unbedingt
globale L�sungen sind.

% TODO?: Konvexit�t

\section{Unrestringierte Optimierungsprobleme}

Wir betrachten nun das unrestringierte Optimierungsproblem
\begin{equation}
  \min_{x \in \R^n} f(x). \label{PU} % TODO: change to (PU) not (1.6)
\end{equation}

Wir nehmen hier der Einfachheit halber an, dass der Definitionsbereich $D$
gleich $\R^n$ sei.

\begin{theorem}
\emph{(Notwendige Bedingung erster Ordnung)}\\
Seien $\xopt$ eine lokale L�sung des Problems~(\ref{PU}) und $f$ einmal stetig
differenzierbar in einer Umgebung von $\xopt$, dann gilt
\begin{equation}
  \nabla f(\xopt) = 0 \label{Gradient ist null}
\end{equation}
\end{theorem}

Diese Bedingung gilt aber nicht nur f�r lokales Minimum sondern auch f�r lokales
Maximum von $f$.

\begin{definition}
\emph{(Station�rer Punkt)}\\
$f$ sei in $\xopt$ differenzierbar. $\xopt$ hei�t ein station�rer Punkt von $f$,
wenn $\xopt$ die notwendige Bedingung~(\ref{Gradient ist null}) erf�llt.
\end{definition}

Viele Optimierungsverfahren suchen normalerweise nach einem station�rem
Punkt von~$f$. Aber ein station�rer Punkt muss nicht ein globales oder lokales Minimum sein.

\begin{theorem}
\emph{(Notwendige Bedingung zweiter Ordnung)}\\
Seien $\xopt$ eine lokale L�sung des Problems~(\ref{PU}) und $f$ zweimal stetig
differenzierbar in einer Umgebung von $\xopt$, dann gilt~(\ref{Gradient ist
null}) und
\begin{equation}
  x^T f''(\xopt) x \geq 0 \qquad \forall x \in \R^n,
\end{equation}
$f''(\xopt)$ ist also positiv semidefinit.
\end{theorem}

Durch diese notwendige Bedingung k�nnen wir zwischen einem lokalen Minimum und
Maximum unterscheiden.

\begin{theorem}
\emph{(Hinreichende Bedingung zweiter Ordnung)}\\
Sei $f$ zweimal stetig differenzierbar in einer Umgebung von $\xopt$.
Die notwendige Bedingung~(\ref{Gradient ist null}) sei erf�llt und $f''(\xopt)$
sei positiv definit, d.h.
\begin{equation}
  x^T f''(\xopt) x > 0 \qquad \forall x \in \R^n.
\end{equation}
Dann ist $\xopt$ eine strikte L�sung des Problems~(\ref{PU}).
\end{theorem}

Diese hinreichende Bedingung benutzt man in der Regel erst dann, wenn man einen
station�ren Punkt findet.

\begin{definition}
\emph{(Abstiegsrichtung)}\\
Sei $f:\R^n \rightarrow \R$ differenzierbar in $x$. Ein Vektor
$d \in \R^n\backslash\{0\}$ hei�t Abstiegsrichtung von~$f$ in~$x$, wenn
\begin{equation}
  \nabla f(x)^T d < 0
\end{equation}
gilt.
\end{definition}

Ist $\nabla f(x) \neq 0$, dann ist beispielsweise $d = - \nabla f(x)$ eine
Abstiegsrichtung von $f$ in $x$.

\begin{theorem}
Seien $f:\R^n \rightarrow \R$ differenzierbar in $x$ und $d$ eine
Abstiegsrichtung von $f$ in $x$. Dann gibt es ein $\hat{\sigma} > 0$ mit
\begin{equation}
  f( x + \sigma d) < f(x) \qquad \forall \sigma \in ]0, \hat{\sigma}[.
\end{equation}
\end{theorem}

Die meisten Optimierungsverfahren sind iterativ. Sie fangen mit einem
Anfangspunkt $x^0$ an und versuchen dann weitere Punkte ($x^1, x^2, \ldots , x^k
, \ldots$) zu finden, die besser als die vorherige sind.
Viele iterative Verfahren zur Bestimmung eine lokale L�sung sind h�ufig
Abstiegsverfahren. In der $k$-ten Iteration bestimmen sie zu einem Punkt $x^k$
eine Abstiegsrichtung $d^k$ und eine Schrittweite $\sigma_k$, sodass f�r
$x^{k+1} := x^k + \sigma_k d^k$
\begin{equation}
  f(x^{k+1}) < f(x^k)
\end{equation}
gilt.

Ein Verfahren, welches die negative Gradienten als Abstiegsrichtungen verwendet,
ist das Gradientenverfahren.

\begin{algorithm}
\emph{(Gradientenverfahren)}
\begin{enumerate}
  \item W�hle einen Startpunkt $x^0$ und ein Abbruchkriterium $\epsilon > 0$.
        Setze $k := 0$.
  \item Ist $||\nabla f(x^k)|| < \epsilon$
        $\Rightarrow$ STOP.
  \item Setze $d^k := - \nabla f(x^k)$.
  \item Bestimme $\sigma_k$ so, dass
        \begin{equation}
          f(x^k + \sigma_k d^k) < f(x^k + \sigma d^k)
            \qquad \forall \sigma \geq 0.
        \end{equation}
  \item Setze $x^{k+1} := x^k + \sigma_k d^k$ und $k := k+1$ $\Rightarrow$
        Gehe zu Schritt 2. %TODO use reference, not 2.
\end{enumerate}
\end{algorithm}

% TODO: Newton-Verfahren

% TODO: Approximation der Gradient und Hesse-Matrix?
