\begin{definition}
Allgemein ist die Aufgabenstellung der Optimierung wie folgt
definiert:
\begin{equation}
  \min_{x \in \F} f(x) \label{eq:op} %use \tag{P} to make (P) as reference
\end{equation}
Die Funktion $f:D \subseteq \R^n \rightarrow \R$ ist die sogennante
Zielfunktion.
$D$ sei der Definitionsbereich von $f$.
$\F$ sei eine nichtleere Teilmenge von $D$, die man als L�sungsmenge bezeichnet.
Alle Elemente von $\F$ werden als zul�ssige Punkte bezeichnet.
$\F$ wird durch die sogennanten Nebenbedingungen definiert.
\end{definition}

Ein einfaches Beispiel ist das Problem
\[
  \min_{x \in \R}\ (x-1)^2.
\]

Falls $\F = D$ gilt,
dann bezeichnet man das Optimierungsproblem als unrestringiert.
Es besitzt also keine Nebenbedingungen.
Ansonsten hei�t es ein restingiertes Optimierungsproblem.

Es ist �blich, dass man nur Minimierungsproblem betrachtet, weil ein
Maximierungsproblem $\max g(x)$ zu dem Minimierungsproblem $\min f(x) := -g(x)$
�quivalent ist.

\begin{definition}
\emph{(Globale und lokale L�sung)}
Ein Punkt $\xopt \in \F$ hei�t globale L�sung des Problems~\eqref{eq:op} oder
globales Minimum von $f$, wenn
\begin{equation}
  f(\xopt) \leq f(x) \qquad \forall x \in \F
\end{equation}
gilt.
Ein Punkt $\xopt \in \F$ hei�t strikte globale L�sung des Problems~\eqref{eq:op} oder
striktes globales Minimum von $f$, wenn
\begin{equation}
  f(\xopt) < f(x) \qquad \forall x \in \F\backslash\{\xopt\}
\end{equation}
gilt.
Ein Punkt $\xopt \in \F$ hei�t lokale L�sung des Problems~\eqref{eq:op} oder
lokales Minimum von~$f$, wenn f�r eine Umgebung $U(\xopt)$ von $\xopt$
\begin{equation}
  f(\xopt) \leq f(x) \qquad \forall x \in U(\xopt) \cap \F
\end{equation}
gilt.
Ein Punkt $\xopt \in \F$ hei�t strikte lokale L�sung des Problems~\eqref{eq:op}
oder striktes lokales Minimum von~$f$, wenn f�r eine Umgebung $U(\xopt)$ von
$\xopt$
\begin{equation}
  f(\xopt) < f(x) \qquad \forall x \in U(\xopt) \cap \F\backslash\{\xopt\}
\end{equation}
gilt.
Eine Umgebung $U(\xopt)$ von $\xopt$ ist einfach eine offene Menge, die $\xopt$
beinhaltet.
\end{definition}

Wegen dieser Definitionen kommt die Unterscheidung zwischen der globalen
und der lokalen Optimierung. Globale Optimierung versucht, globale L�sungen
zu finden. Lokale Optimierung versucht dagegen, lokale L�sungen zu finden.
Viele Verfahren finden lokale L�sungen, die jedoch nicht unbedingt
globale L�sungen sind. Globale L�sungen sind nicht so einfach zu finden.

\begin{definition}
Das allgemeine nichtlineare Optimierungsproblem ist wie folgt definiert:
\begin{align}
       \min\ & f(x) \\
         \nb & g(x) \leq 0 \\
             & h(x) = 0
\end{align}
Die Zielfunktion ist wieder die Funktion $f:D_f \subseteq \R^n \rightarrow \R$.
Die Nebenbedingungen sind von den Funktionen
$g:D_g \subseteq \R^n \rightarrow \R^p$ und
$h:D_h \subseteq \R^n \rightarrow \R^m$
abh�ngig.
\end{definition}

D.h. die Menge $\F$ sieht hier so aus:
\begin{equation}
  \F := \{ x | g(x) \leq 0, h(x) = 0  \}.
\end{equation}

Man kann hierbei den Unterschied zwischen der linearen Optimierung und
der nichtlinearen Optimierung gut erkennen.
Bei der linearen Optimierung muss die Zielfunktion linear sein
(d.h. die Zielfunktion besitzt die Form $f(x) = c^T x$, $c \in \R^n$)
und die Nebenbedingungen sind durch lineare Gleichungssysteme oder
Ungleichungssyteme definiert.
Bei der nichtlinearen Optimierung gibt es dagegen keine Einschr�nkung,
wie die Zielfunktion und die Nebenbedingungen aussehen sollen.
