\begin{definition}
\emph{Optimierungsproblem mit linearen Ungleichungsnebenbedingungen}
\begin{align}
  \min_{x \in \R^n}\ & f(x) \label{eq:opmlugnb} \\
                 \nb & Ax = b \notag \\
                     & Gx = r \notag
\end{align}
$A$ sei eine $(m \times n)$-Matrix und $b$ sei ein Vektor mit $m$ Elementen.
$G$ sei eine $(p \times n)$-Matrix und $r$ sei ein Vektor mit $p$ Elementen.
\end{definition}

D.h. die Menge $\F$ sieht hier so aus:
\begin{equation}
  \F := \{ x | A x = b, G x \leq r \}.
\end{equation}

\begin{theorem}
\emph{(Karush-Kuhn-Tucker-Satz)}
Sei $\xopt$ lokale L�sung des Problems~\eqref{eq:opmlugnb} und
$f$ sei in~$\xopt$ differenzierbar.
Dann existieren die Vektoren $\lambda \in \R^m$ und $\mu \in \R^p$
zu~$\xopt$ mit
\begin{gather}
  \nabla f(\xopt) + A^T \lambda + G^T \mu = 0 \\
  \mu \geq 0, \quad \mu_j (<g^j,\xopt> - r_j ) = 0, \quad j = 1,\ldots,p.
\end{gather}
$\lambda$ und $\mu$ hei�en Lagrange-Multiplikatoren zu $\xopt$.
\end{theorem}
