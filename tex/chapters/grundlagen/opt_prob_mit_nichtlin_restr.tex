Wir kommen nun zu dem allgemeinen nichtlinearen Optimierungsproblem.
Wir schreiben noch einmal das Problem~\eqref{prob:opt_prob_mit_nichtlin_restr}.
\begin{align*}
  \min_{x \in \R^n}\ & f(x) \\
              \nb & g_i(x) \leq 0 \text{ f�r } i = 1,\ldots,p \\
                  & h_j(x) = 0 \text{ f�r } j = 1,\ldots,m
\end{align*}

Um die Optimalit�tsbedingung des Problems~\eqref{prob:opt_prob_mit_nichtlin_restr}
zu bekommen, definiert man die sogenannte Regularit�tsbedingung
(vgl. Definition 7.2.14 in \cite[S.~261]{alt}),
die wir hier nicht n�her betrachten werden.
Die folgende Definition ist eine spezielle Bedingung,
aus der diese Regularit�tsbedingung folgt.
Diese Regularit�tsbedingung kann aber
noch von anderen Bedingungen hergeleitet werden.

\begin{definition}
\emph{(Mangasarian-Fromowitz-Regularit�tsbedingung, vgl. Satz 7.2.24 in
\cite[S.~270]{alt})}
Sei $x \in \F$. Mit
$\,\I(x) := \{ i \in \{1,\ldots,p\} \ | \ g_i(x) = 0 \}$
wird die Indexmenge der in $x$ aktiven Ungleichungsrestriktionen bezeichnet.
Seien $g_i$ und $h_j$ differenzierbar in~$x$
f�r $i=1,\ldots,p$ und $j=1,\ldots,m$.
Der Punkt $x$ erf�llt die Mangasarian-Fromowitz-Regularit�tsbedingung,
wenn die Gradienten
\begin{equation}
 \nabla h_j(x),\ j = 1,\ldots,m, \text{ linear unabh�ngig}
\end{equation}
sind und ein Vektor $d \in \R^n$ mit
\begin{equation}
  \nabla g_i(x)^T d < 0,\ i \in \I(x)
  \quad \text{und} \quad
  \nabla h_j(x)^T d = 0,\ j = 1,\ldots,m
\end{equation}
existiert.
\end{definition}

Der folgende Satz gilt eigentlich mit der allgemeinen Regularit�tsbedingung.
Wir verwenden aber hier nur die Mangasarian-Fromowitz-Regularit�tsbedingung.

\begin{theorem}
\emph{(Notwendige Bedingung, vgl. Satz 7.2.11 in \cite[S.~260]{alt})}\\
Sei $\xopt$ eine lokale L�sung des
Problems~\eqref{prob:opt_prob_mit_nichtlin_restr} und $\xopt$ erf�lle die
Mangasarian-Fromowitz-Regularit�tsbedingung.
Sei $f$ in $\xopt$ differenzierbar.
Dann existieren Vektoren $\lambda \in \R^m$ und $\mu \in \R^p$,
sodass
\begin{gather}
  \nabla f(\xopt) + \sum_{i=1}^{m} \lambda_i \nabla h_i(\xopt)
    + \sum_{j=1}^{p} \mu_j \nabla g_j(\xopt) = 0 \\
  \mu_j \geq 0, \quad \mu_j g_j(\xopt) = 0, \quad
  j=1,\ldots,p
\end{gather}
\end{theorem}

F�r die hinreichende Bedingung zweiter Ordnung verweisen wir auf Satz 7.3.1
in \cite[S.~273]{alt}.