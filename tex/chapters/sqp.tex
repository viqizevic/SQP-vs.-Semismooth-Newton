\chapter{Sequentielle Quadratische Programmierung}
\section{Einf�hrung}

Sequentielle Quadratische Optimierung ist ein bekanntes Verfahren, um ein
Problem mit nicht linearer Zielfunktion und linearen Nebenbedigungen zu l�sen.


\begin{figure}[htbp]
\centering
\includegraphics[width=0.7\textwidth]{TUBerlin_Logo} % Datei in "bilder/" bei
% LaTeX: eps, bei PDFLaTeX: jpg (o.�.)
\caption{Titel der Abbildung} 
\label{fig:bild1}
\end{figure}

In der Abbildung~\ref{fig:bild1}\footnote{vgl. Zitat A\cite{ItoKun}} %
% referenzA muss im Verzeichnis literatur in der Datei bib.bib enthalten sein
ist zu sehen, dass ...

% Neue Seite
\newpage

\section{Und n�chster Abschnitt}
Eine neue Seite, um auchmal die Kopfzeile zu sehen, da sie auf Seiten mit Kapitelanfang nicht erscheinen
