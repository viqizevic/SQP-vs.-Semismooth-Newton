\chapter{SQP-Verfahren}
\section{Einf�hrung}

Ein SQP-Verfahren ist ein wichtiges Verfahren, um restringierte
nichlineare Probleme zu l�sen. SQP ist eine Abk�rzung f�r Sequentielle
Quadratische Optimierung.

Erstes: Wilson im Jahr 1963
% [120] Wilson R. B.: A simplical algorithm for concave programming. PhD Thesis.
% Harvard university, Boston 1963.
% [92] Han und Powell: Algorithms 
% [41]

Die Idee: Iterativ werden durch das L�sen von quadratischen Problemen KKT-Punkte
gefunden, also Kandidaten f�r lokale Minimalstellen.

Und zum L�sen des quadratischen Problems ist ein effizientes Verfahren bekannt.


\begin{figure}[htbp]
\centering
\includegraphics[width=0.7\textwidth]{TUBerlin_Logo} % Datei in "bilder/" bei
% LaTeX: eps, bei PDFLaTeX: jpg (o.�.)
\caption{Titel der Abbildung} 
\label{fig:bild1}
\end{figure}

In der Abbildung~\ref{fig:bild1}\footnote{vgl. Zitat A\cite{ItoKun}} %
% referenzA muss im Verzeichnis literatur in der Datei bib.bib enthalten sein
ist zu sehen, dass ...

% Neue Seite
\newpage

\section{Und n�chster Abschnitt}
Eine neue Seite, um auchmal die Kopfzeile zu sehen, da sie auf Seiten mit Kapitelanfang nicht erscheinen
