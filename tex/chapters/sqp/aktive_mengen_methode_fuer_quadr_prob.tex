
Wir haben gesehen, dass das SQP-Verfahren
quadratische Optimierungsprobleme mit linearen Restriktionen
als Teilprobleme hat.
Wir werden jetzt aufkl�ren, wie wir diese Probleme l�sen werden.

\begin{definition}
\emph{(Quadratische Optimierungsprobleme mit linearen Restriktionen)}
\begin{align}
  \min_{x \in \R^n}\ &
    \frac{1}{2} \langle Qx,x \rangle + \langle q,x \rangle 
    \tag{QLG} \label{prob:quad_prog_mit_lin_restr} \\
  \nb & Ax = b \notag \\
      & Gx \leq r \notag
\end{align}
Dabei sei $Q$ eine symmetrische $(n \times n)$-Matrix und $q \in \R^n$.
$A$ sei eine $(m \times n)$-Matrix mit $m \leq n$ und $b \in \R^m$.
$G$ sei eine $(p \times n)$-Matrix und $r \in \R^p$.
\end{definition}

Sei $x^k \in \R^n$ ein Punkt in dem Iterationschritt.
Seien $J(x^k)$ die Indexmenge der aktiven Ungleichungsrestriktionen zu $x^k$
und $p_k := |J(x^k)|$. Seien
\begin{equation}
G_k :=
\left(\begin{array}{c}
g_{j_1}^T \\ \vdots \\ g_{j_{p_k}}^T
\end{array}\right) \text{ mit }
j_1, \ldots, j_{p_k} \in J(x^k)
\qquad \text{und} \qquad
B_k := \left(\begin{array}{c} A \\ G_k \end{array}\right)
\end{equation}

\begin{align}
  \min_{d \in \R^n}\ & \frac{1}{2} \langle Qd,d \rangle
                       + \langle Qx^k+q,d\rangle
                         \tag{$Q_k$} \label{prob:teil_prob_in_akt_me_verf} \\
                 \nb & B_k d = 0 \notag
\end{align}

\begin{algorithm}
\emph{(Aktive-Mengen-Methode f�r~\eqref{prob:quad_prog_mit_lin_restr})}
\begin{compactenum}[1.]
  \item W�hle einen Startpunkt $x^0$. Setze $k:=0$.
  \item L�se das Problem~\eqref{prob:teil_prob_in_akt_me_verf}
        $\Rightarrow d^k, \lambda^k, \mu^k$.
  \item Falls $d^k = 0$
    \begin{compactenum}[3a.]
      \item Falls $\mu^k \geq 0 \Rightarrow$ STOP.
      \item Falls $\mu^k \ngeq 0$
        \begin{compactenum}[i.]
          \item Bestimme $j \in J(x^k)$, so dass
                $\mu_j^k = \min\{ \mu_i^k \ |\ i \in J(x^k) \}$.
          \item $J(x^k) := J(x^k)\backslash\{j\}$ 
          \item Streiche in $G_k$ die $j$-te Zeile.
          \item L�se wieder das Problem~\eqref{prob:teil_prob_in_akt_me_verf}
                $\Rightarrow d^k \neq 0$
        \end{compactenum}
    \end{compactenum}
  \item Berechne Schrittweite $\sigma_k$
  \item Setze $x^{k+1} := x^k + \sigma_k d^k$ und $k := k+1$.
        $\Rightarrow $ Gehe zu Schritt 2. %FIMXE use label
\end{compactenum}
\end{algorithm}