
Wir haben gesehen, dass das SQP-Verfahren
quadratische Optimierungsprobleme mit linearen Restriktionen
als Teilprobleme hat.
Wir werden jetzt aufkl�ren, wie wir diese Probleme
mit dem Aktive-Mengen-Verfahren l�sen k�nnen.

\begin{definition}
\emph{(Quadratische Optimierungsprobleme mit linearen Restriktionen)}
\begin{align}
  \min_{x \in \R^n}\ &
    \frac{1}{2} \langle Qx,x \rangle + \langle q,x \rangle 
    \tag{QLU} \label{prob:quad_prog_mit_lin_restr} \\
  \nb & Ax = b \notag \\
      & Gx \leq r \notag
\end{align}
Dabei sei $Q$ eine symmetrische $(n \times n)$-Matrix und $q \in \R^n$.
$A$ sei eine $(m \times n)$-Matrix mit $m \leq n$ und $b \in \R^m$.
$G$ sei eine $(p \times n)$-Matrix und $r \in \R^p$.
\end{definition}

Die Idee des Aktive-Mengen-Verfahren ist, iterativ das Problem
unter den Gleichungsnebenbedingungen zu optimieren.
%TODO noch unklar

Sei $x^k \in \R^n$ ein zul�ssiger Punkt.
Angenommen, wir k�nnen eine Suchrichtung $d$ finden,
sodass der Punkt $x^{k+1} := x^k + d$ ein besserer zul�ssiger Punkt ist.
Diesen Vektor $d$ k�nnen wir finden, indem wir das Problem
\begin{align}
  \min_{d \in \R^n}\ &
    \frac{1}{2} \langle Q(x^k+d),(x^k+d) \rangle + \langle q,(x^k+d) \rangle 
      \\
  \nb & A(x^k+d) = b \notag \\
      & G(x^k+d) \leq r \notag
\end{align}
l�sen.
Dies ist eigentlich das Problem~\eqref{prob:quad_prog_mit_lin_restr},
wobei die Variable $x$ mit $x^k + d$ ersetzt wurde und das Problem �ber
die Variable $d$ minimiert wurde.

Lassen wir die Konstanten in der Zielfunktion weg, dann bekommen wir
als Zielfunktion
\begin{displaymath}
  \frac{1}{2} \langle Qd,d \rangle + \langle Qx^k+q,d\rangle.
\end{displaymath}
Weil $x^k$ zul�ssig ist, d.\,h., es gilt $Ax^k = b$, k�nnen wir wieder als
Gleichungsnebenbedingung $Ad = 0$ schreiben.

Seien $J(x^k)$ die Indexmenge der aktiven Ungleichungsrestriktionen zu $x^k$.
Sei $j \in J(x^k)$,
d.\,h., es gilt $\langle g_j,x^k \rangle = r_j$.
Setzen wir voraus, dass
$\langle g_j,d \rangle = 0$, dann gilt immer noch
$\langle g_j,(x^k+d) \rangle \leq r_j$.

Seien $p_k := |J(x^k)|$,
\begin{equation}
G_k :=
\left(\begin{array}{c}
g_{j_1}^T \\ \vdots \\ g_{j_{p_k}}^T
\end{array}\right) \text{ mit }
j_1, \ldots, j_{p_k} \in J(x^k)
\quad \text{und} \quad
B_k := \left(\begin{array}{c} A \\ G_k \end{array}\right)
\end{equation}

Wir haben dann ein Teilproblem mit linearen Gleichungsnebenbedingungen
\begin{align}
  \min_{d \in \R^n}\ & \frac{1}{2} \langle Qd,d \rangle
                       + \langle Qx^k+q,d\rangle
                         \tag{$Q_k$} \label{prob:teil_prob_in_akt_me_verf} \\
                 \nb & B_k d = 0. \notag
\end{align}

Dieses Problem werden wir mit dem Nullraum-Verfahren l�sen,
von dem wir auch den Lagrange-Multiplikator $\vartheta$
bekommen werden.
$\vartheta$ k�nnen wir in $\lambda \in \R^m$ und $\mu \in \R^{p_k}$ zerlegen,
sodass
$\vartheta = \left(\begin{array}{c} \lambda \\ \mu \end{array}\right)$.

\begin{algorithm}
\emph{(Aktive-Mengen-Methode f�r~\eqref{prob:quad_prog_mit_lin_restr})}
\begin{compactenum}[1.]
  \item W�hle einen Startpunkt $x^0$. Setze $k:=0$.
  \item L�se das Problem~\eqref{prob:teil_prob_in_akt_me_verf}
        $\Rightarrow d^k, \lambda^k, \mu^k$.
  \item Falls $d^k = 0$
    \begin{compactenum}[3a.]
      \item Falls $\mu^k \geq 0 \Rightarrow$ STOP.
      \item Falls $\mu^k \ngeq 0$
        \begin{compactenum}[i.]
          \item Bestimme $j \in J(x^k)$, so dass
                $\mu_j^k = \min\{ \mu_i^k \ |\ i \in J(x^k) \}$.
          \item $J(x^k) := J(x^k)\backslash\{j\}$ 
          \item Streiche in $G_k$ die $j$-te Zeile.
          \item L�se wieder das Problem~\eqref{prob:teil_prob_in_akt_me_verf}
                $\Rightarrow d^k \neq 0$
        \end{compactenum}
    \end{compactenum}
  \item Berechne Schrittweite $\sigma_k$.
  \item Setze $x^{k+1} := x^k + \sigma_k d^k$ und $k := k+1$.
        $\Rightarrow $ Gehe zu Schritt 2. %FIMXE use label
\end{compactenum}
\end{algorithm}

%TODO Schrittweite erkl�ren
\begin{equation}
  I_k := \{ 1, \ldots, n \} \backslash J_k
\end{equation}
\begin{equation}
  \tau_k :=
  \begin{cases}
    \min\left\{ \frac{r_j - g_j^T x^k}{g_j^T d^k}\ | \ j \in I_k \right\},
      & \text{falls } I_k \neq \emptyset \\
    \infty,
      & \text{sonst}.
  \end{cases}
\end{equation}
\begin{equation}
  \sigma_k := \min \{ 1, \tau_k\}
\end{equation}

Die Schrittweite brauchen wir, damit $x^{k+1}$ ein zul�ssiger Punkt bleibt.