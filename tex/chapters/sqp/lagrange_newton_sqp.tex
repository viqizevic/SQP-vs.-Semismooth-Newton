Nun kommen wir zu den allgemeinen nichtlinearen Optimierungsproblemen.

\begin{definition}
Wir definieren die Lagrange-Funktion
$\Lagrange : D \times \R^m \times \R^p \rightarrow \R$ durch
\begin{equation}
\Lagrange(x,\lambda,\mu) := f(x) + \lambda^T h(x) + \mu^T g(x).
\end{equation}
\end{definition}

\begin{algorithm}
\emph{(Lagrange-Newton-SQP-Verfahren, vgl. Verfahren 8.1.4 in
\cite[S.~294]{alt})}
\begin{enumerate}
  \item W�hle einen Startpunkt $z^0 := (x^0, \lambda^0, \mu^0)$
        und $\epsilon > 0$.
        Setze $k := 0$.
  \item Berechne die Suchrichtung $d^k$ durch L�sung des Problems
    \begin{align}
      \min_{d \in \R^n}\ &
        \frac{1}{2} d^T \Lagrange_{xx}(x^k,\lambda^k,\mu^k) d
          + \nabla f(x^k)^T d \\
      \nb & h(x^k) + h'(x^k) d = 0 \\
          & g(x^k) + g'(x^k) d \leq 0
    \end{align}
  \item Ist $\|d^k\| < \epsilon \Rightarrow$ STOP.
  \item Setze $x^{k+1} := x^k + d^k$ und $k := k+1$.
        $\Rightarrow$ Gehe zu Schritt 2.
\end{enumerate}
\end{algorithm}