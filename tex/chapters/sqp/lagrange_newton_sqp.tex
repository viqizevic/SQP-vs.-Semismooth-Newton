Nun kommt das Verfahren
f�r die allgemeinen nichtlinearen Optimierungsprobleme.

\begin{definition}
Die Lagrange-Funktion
$\Lagrange : D \times \R^m \times \R^p \rightarrow \R$
des Problems~\eqref{prob:opt_prob_mit_nichtlin_restr}
sei durch
\begin{equation}
\Lagrange(x,\lambda,\mu) := f(x) + \lambda^T h(x) + \mu^T g(x)
\end{equation}
definiert.
\end{definition}

\begin{algorithm}
\emph{(Lagrange-Newton-SQP-Verfahren, vgl. Verfahren 8.1.4 in
\cite[S.~294]{alt})}
\begin{enumerate}
  \item W�hle einen Startpunkt $z^0 := (x^0, \lambda^0, \mu^0)$
    und setze $k := 0$.
  \item L�se das Problem
    \begin{align}
      \min_{d \in \R^n}\ &
        \frac{1}{2} d^T \Lagrange_{xx}(x^k,\lambda^k,\mu^k) d
          + \nabla f(x^k)^T d \\
      \nb & h(x^k) + h'(x^k) d = 0 \\
          & g(x^k) + g'(x^k) d \leq 0.
    \end{align}
    $\Rightarrow$ Erhalte $d, \lambda$ und $\mu$.
  \item Ist $d = 0 \Rightarrow$ STOP.
  \item Setze $x^{k+1} := x^k + d,\ \lambda^{k+1} := \lambda,\ 
    \mu^{k+1} := \mu$ und $k := k+1$.
    $\Rightarrow$ Gehe zu Schritt 2.
\end{enumerate}
\end{algorithm}

In diesem Verfahren wird also iterativ die Suchrichtung
�ber eine quadratische N�herung der Lagrange-Funktion
des Problems berechnet,
wobei die Nebenbedingungen linearisiert werden.

Eine ausf�hrliche Erkl�rung und ein Konvergenzresult�t zu diesem Verfahren
sind in Abschnitt~8.1.2 in \cite[S.~290f]{alt} zu finden.