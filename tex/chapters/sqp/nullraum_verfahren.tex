Wie das SQP-Verfahren die Hilfe von Aktive-Mengen-Verfahren braucht,
braucht das Aktive-Mengen-Verfahren die Hilfe von Nullraum-Verfahren,
um die entstehenden quadratischen Teilprobleme mit nur linearen
Gleichungsnebenbedingungen zu l�sen.

%-------------------------------------------------------------------------%
%                                                                         %
% Definition (Quadr. Opt.-Prob. mit lin. Gl.-Nebenbed.)                   %
%                                                                         %
%-------------------------------------------------------------------------%

\begin{definition}
\emph{(Quadratische Optimierungsprobleme mit linearen
Gleichungsnebenbedingungen)}
\begin{align}
  \min_{x \in \R^n}\ & f(x) := \frac{1}{2} \langle Qx,x \rangle
    + \langle q,x \rangle \tag{QLG}
    \label{prob:quad_opt_prob_mit_lin_gl_nebenbed} \\
  \nb & Ax = b \notag
\end{align}
Dabei sei $Q$ eine symmetrische $(n \times n)$-Matrix und $q \in \R^n$.
$A$ sei eine $(m \times n)$-Matrix mit $m \leq n$ und $b \in \R^m$.
\end{definition}

Das Problem~\eqref{prob:quad_opt_prob_mit_lin_gl_nebenbed} wird hier auf ein
unrestringiertes Verfahren mit Hilfe einer Nullraum-Matrix von~$A$
(siehe Definition~\ref{def:nullraum_matrix}) reduziert werden.
Daher stammt der Name \emph{Nullraum-Verfahren}.
Es folgen zun�chst s�mtliche Schritte des Verfahrens und
anschlie�end die Erkl�rungen der einzelnen Schritte.

%-------------------------------------------------------------------------%
%                                                                         %
% Algo (Nullraum-Verf. f. quadr. Opt.-Prob. mit lin. Gl.-Nebenbed.)       %
%                                                                         %
%-------------------------------------------------------------------------%

\begin{algorithm}
\emph{(Nullraum-Verfahren f�r \eqref{prob:quad_opt_prob_mit_lin_gl_nebenbed},
vgl. Verfahren 6.1.1 in \cite[S.~208~f.]{alt})}
\begin{enumerate}
  \item Finde mit Hilfe der QR-Zerlegung
  eine unit�re $(n \times n)$-Matrix~$H$
  und eine obere $(m \times m)$-Dreiecksmatrix~$R$ mit
  \begin{equation}
    H A^T = \left(\begin{array}{c} R \\ 0 \end{array}\right).
  \end{equation}
  \label{list:qr_zerlegung_in_nullraum_verfahren}
  
  \item Berechne den Vektor $y \in \R^m$ als L�sung der Gleichung
  \begin{equation}
    R^T y = b.
  \end{equation}
  \label{list:vektor_y_in_nullraum_verfahren}
  
  \item Berechne
  \begin{equation}
    h := -Hq
       = \left(\begin{array}{c} h_1 \\ h_2 \end{array}\right)
    \quad \text{und} \quad
    B := H Q H^T
       = \left(\begin{array}{cc} B_{11} & B_{12} \\ B_{21} & B_{22}
         \end{array}\right),
  \end{equation}
  wobei $h_1 \in \R^m$ und $B_{11} \in \R^{m,m}$.
  \label{list:matrix_B_in_nullraum_verfahren}
  
  \item Berechne den Vektor $z \in \R^{n-m}$ als L�sung der Gleichung
  \begin{equation}
    B_{22} z = h_2 - B_{21} y.
  \end{equation}
  \label{list:vektor_z_in_nullraum_verfahren}
  
  \item Berechne $\xopt$ mit
  \begin{equation}
    \xopt := H^T \left(\begin{array}{c} y \\ z \end{array}\right).
  \end{equation}
  \label{list:opt_loesung_in_nullraum_verfahren}
  
  \item Der Multiplikator $\lambda$ ist die L�sung der Gleichung
  \begin{equation}
    R \lambda = h_1 - B_{11} y - B_{12} z.
  \end{equation}
  \label{list:multiplikator_lambda_in_nullraum_verfahren}
\end{enumerate}
\end{algorithm}

%-------------------------------------------------------------------------%
% Erkl�rung: In der Matrix H ist die Nullraum-Matrix Z gespeichert
%

In Schritt~\ref{list:qr_zerlegung_in_nullraum_verfahren}
bekommt man die Matrizen $H$ und $R$.
Die Matrix $H$ kann so zerlegt werden, dass
\begin{equation}
  H^T = \left(\begin{array}{cc} Y & Z \end{array}\right)
\end{equation}
mit $Y \in \R^{n,m}$ und $Z \in \R^{n,n-m}$ gilt.
Wegen der Orthogonalit�t von H
k�nnen die Spaltenvektoren in $Y$ und $Z$
eine Basis von $\R^n$ bilden.
D.\,h., f�r jedes Element $x \in \R^n$
gibt es eindeutig bestimmte Vektoren
$x_y \in R^m$ und $x_z \in \R^{n-m}$, sodass
\begin{equation}
  x = Y x_y + Z x_z = H^T \left(\begin{array}{c} x_y \\ x_z \end{array}\right)
\end{equation}
gilt.
Sei $d \in \kr{A} \subseteq \R^n$.
Dann gibt es $d_y \in R^m$ und $d_z \in \R^{n-m}$ mit $d = Y d_y + Z d_z$.
Es gilt folglich
\begin{equation}
Ad = AY d_y + AZ d_z = AH^T \left(\begin{array}{c} d_y \\ d_z \end{array}\right)
   = (R^T \ 0) \left(\begin{array}{c} d_y \\ d_z \end{array}\right)
   = R^T d_y
\end{equation}
und wegen der Invertierbarkeit der Matrix $R$
\begin{equation}
  Ad = 0 \quad \Leftrightarrow \quad d_y = 0
  \quad \Leftrightarrow \quad d = Z d_z.
\end{equation}
Daraus folgt
\begin{equation}
  \kr{A} = \{ d \in \R^n\, |\, Ad = 0 \}
    = \{ d \in \R^n\, |\, \exists d_z \in R^{n-m} : d = Z d_z \} = \im{Z}.
\end{equation}
Deshalb ist diese Matrix $Z$ die Nullraum-Matrix von $A$.

%-------------------------------------------------------------------------%
% Erkl�rung: Wie eine spezielle L�sung der Gleichung Ax = b zu finden ist
%

Sei $\xopt$ die optimale L�sung. Dann gibt es $y^*$ und $z^*$ mit
\begin{equation}
\label{eq:zerlegung_von_xopt_in_nullraum_verfahren}
  \xopt = Y y^* + Z z^*
    = H^T \left(\begin{array}{c} y^* \\ z^* \end{array}\right).
\end{equation}
Es gilt
\begin{equation}
  b = A \xopt
    = A H^T \left(\begin{array}{c} y^* \\ z^* \end{array}\right)
    = \left(\begin{array}{cc} R^T & 0 \end{array}\right)
      \left(\begin{array}{c} y^* \\ z^* \end{array}\right)
    = R^T y^*
\end{equation}
Daher ist in
Schritt~\ref{list:vektor_y_in_nullraum_verfahren}
eigentlich der Vektor $y^*$ zu bestimmen.

Da $Z$ eine Nullraum-Matrix ist,
gilt $Zz^* \in \kr{A}$, d.\,h. $AZz^* = 0$.
Dann erf�llt $w := Y y^*$
\begin{equation}
  Aw = Aw + AZz^* = A (w + Zz^*)
     = A (Yy^* + Zz^*) = A \xopt = b
\end{equation}
Der Vektor $w$ ist somit eine spezielle L�sung der Gleichung $Ax=b$.

%-------------------------------------------------------------------------%
% Erkl�rung: Umformulierung des Problems als ein unrestringiertes
%

Nun ergibt sich f�r unser Problem~\eqref{prob:quad_opt_prob_mit_lin_gl_nebenbed}
wie in~\eqref{prob:opt_prob_unrestr_mithilfe_nullraum_matrix}
folgende Umformulierung:
\begin{equation}
  \min_{z \in \R^{n-m}} f(w + Zz) =: F(z).
\end{equation}
Die L�sung dieses Problems erf�llt
nach Satz~\ref{satz:notw_bed_fuer_unrestr_prob}
die Optimalit�tsbedingung
\begin{equation}
  \nabla F(z^*) = 0.
\end{equation}
Es ist
\begin{align}
  F(z) & = f(w + Zz) = \frac{1}{2}(w+Zz)^T Q (w+Zz) + q^T(w+(Zz)) \\
       & = \frac{1}{2}w^T Qw + \frac{1}{2}w^T Q Zz
         + \frac{1}{2}z^T ZQw + \frac{1}{2}z^T Z^T QZz + q^T w+ q^T Zz \\
       & = \frac{1}{2}z^T Z^T QZz + w^T Q Zz + q^T Zz + f(w) \\
       & = \frac{1}{2}z^T Z^T QZz + (Z^T Qw + Z^T q)^T z + f(w) \\
       & = \frac{1}{2}z^T Z^T QZz + (Z^T Q Y y^* + Z^T q)^T z + f(Y y^*).
\end{align}
F�r den Gradienten von $F$ gilt dann
\begin{equation}
  \nabla F(z) = Z^T QZz + Z^T Q Y y^* + Z^T q.
\end{equation}

In Schritt~\ref{list:matrix_B_in_nullraum_verfahren} sind
\begin{equation}
\left(\begin{array}{c} h_1 \\ h_2 \end{array}\right)
  = -Hq
  = - \left(\begin{array}{c} Y^T \\ Z^T \end{array}\right) q
  = - \left(\begin{array}{c} Y^T q \\ Z^T q \end{array}\right)
\end{equation}
und
\begin{equation}
\left(\begin{array}{cc} B_{11} & B_{12} \\ B_{21} & B_{22} \end{array}\right)
  = H Q H^T
  = \left(\begin{array}{c} Y^T \\ Z^T \end{array}\right) Q
      \left(\begin{array}{cc} Y & Z \end{array}\right)
  = \left(\begin{array}{cc} Y^T Q Y & Y^T Q Z \\
      Z^T Q Y & Z^T Q Z \end{array}\right).
\end{equation}

%-------------------------------------------------------------------------%
% Erkl�rung: L�sung des Problems
%

Die Bedingung $\nabla F(z^*) = 0$ kann daher als
\begin{equation}
  B_{22} z^* + B_{21} y^* - h_2 = 0
\end{equation}
geschrieben werden.
Mit Hilfe dieser Gleichung bestimmt man in
Schritt~\ref{list:vektor_z_in_nullraum_verfahren}
den Vektor $z^*$
und folglich in
Schritt~\ref{list:opt_loesung_in_nullraum_verfahren}
die L�sung $\xopt$
nach~\eqref{eq:zerlegung_von_xopt_in_nullraum_verfahren}.

%-------------------------------------------------------------------------%
% Erkl�rung: Berechnung des Multiplikators
%

Zum Abschluss wird in
Schritt~\ref{list:multiplikator_lambda_in_nullraum_verfahren}
der Multiplikator $\lambda$ berechnet.
Nach Satz~\ref{satz:notw_bed_fuer_prob_mit_lin_gl_nebenbed}
lautet die Optimalit�tsbedingung des
Problems~\eqref{prob:quad_opt_prob_mit_lin_gl_nebenbed}
\begin{equation}
  Q \xopt + q + A^T \lambda = 0.
\end{equation}
Multipliziere diese mit $H$, so ergibt sich
\begin{equation}
  0 = HQ \xopt + Hq + H A^T \lambda
    = \left(\begin{array}{c} Y^T \\ Z^T \end{array}\right) Q \xopt
        - \left(\begin{array}{c} h_1 \\ h_2 \end{array}\right)
        + \left(\begin{array}{c} R \\ 0 \end{array}\right) \lambda.
\end{equation}
Aus dem oberen Teil dieses Gleichungssystems l�sst sich die Gleichung
\begin{equation}
  0 = Y^T Q (Y y^* + Z z^*) - h_1 + R^T \lambda
    = B_{11} y^* + B_{12} z^* - h_1 + R^T \lambda
\end{equation}
zur Berechnung von $\lambda$ finden.