Wie das SQP-Verfahren die Hilfe von Aktive-Mengen-Verfahren braucht,
braucht das Aktive-Mengen-Verfahren die Hilfe eines anderen Verfahrens,
um die entstehenden quadratischen Teilprobleme mit nur linearen
Gleichungsnebenbedingungen zu l�sen.

\begin{definition}
\emph{(Quadratische Optimierungsprobleme mit linearen
Gleichungsnebenbedingungen)}
\begin{align}
  \min_{x \in \R^n}\ & f(x) := \frac{1}{2} \langle Qx,x \rangle
    + \langle q,x \rangle \tag{QLG}
    \label{prob:quad_opt_prob_mit_lin_gl_nebenbed} \\
  \nb & Ax = b \notag
\end{align}
Dabei sei $Q$ eine symmetrische $(n \times n)$-Matrix und $q \in \R^n$.
$A$ sei eine $(m \times n)$-Matrix mit $m \leq n$ und $b \in \R^m$.
\end{definition}

Das Problem~\eqref{prob:quad_opt_prob_mit_lin_gl_nebenbed} werden wir auf ein
unrestringiertes Verfahren reduzieren mit Hilfe einer Nullraum-Matrix von~$A$
(siehe Definition~\ref{def:nullraum_matrix}).
Daher kommt der Name \emph{Nullraum-Verfahren}.
Wir schreiben nun zun�chst alle Schritten des Verfahrens.
Anschlie�end kommen die Erkl�rungen des einzelnen Schrittes.

\begin{algorithm}
\emph{(Nullraum-Verfahren f�r \eqref{prob:quad_opt_prob_mit_lin_gl_nebenbed},
vgl. Verfahren 6.1.1 in \cite[S.~208f]{alt})}
\begin{enumerate}
  \item Finde mit Hilfe der QR-Zerlegung
  unit�re $(n \times n)$-Matrix~$H$
  und obere $(m \times m)$-Dreiecksmatrix~$R$ mit
  \begin{equation}
    H A^T = \left(\begin{array}{c} R \\ 0 \end{array}\right).
  \end{equation}
  \label{list:qr_zerlegung_in_nullraum_verfahren}
  \item Berechne
  \begin{equation}
    \left(\begin{array}{c} h_1 \\ h_2 \end{array}\right) := -Hq
    \quad \text{und} \quad
    B := H Q H^T
       = \left(\begin{array}{cc} B_{11} & B_{12} \\ B_{21} & B_{22}
         \end{array}\right),
  \end{equation}
  wobei $h_1 \in \R^m$ und $B_{11} \in \R^{m,m}$
  \item Berechne den Vektor $u_1$ als L�sung der Gleichung
  \begin{equation}
    R^T u_1 = b
  \end{equation}
  und den Vektor $u_2$ als L�sung der Gleichung
  \begin{equation}
    B_{22} u_2 = h_2 - B_{21} u_1
  \end{equation}
  \item Berechne $\xopt$
  \begin{equation}
    \xopt := H^T \left(\begin{array}{c} u_1 \\ u_2 \end{array}\right)
  \end{equation}
  \item Der Multiplikator $\lambda^*$ ist die L�sung der Gleichung
  \begin{equation}
    R \lambda^* = h_1 - B_{11} u_1 - B_{12} u_2
  \end{equation}
\end{enumerate}
\end{algorithm}

Bei Schritt~\ref{list:qr_zerlegung_in_nullraum_verfahren}
berechnen wir die Matrizen $H$ und $R$.
Die Matrix $H$ kann so zerlegt werden, dass
\begin{equation}
  H^T = \left( Y \ Z \right)
\end{equation}
mit $Y \in \R^{n,m}$ und $Z \in \R^{n,n-m}$.
Wegen der Orthogonalit�t von H
k�nnen die Spaltenvektoren in $Y$ und $Z$
eine Basis von $\R^n$ bilden.
D.\,h., f�r jedes Element $x \in \R^n$
gibt es eindeutig bestimmte Vektoren
$x_y \in R^m$ und $x_z \in \R^{n-m}$, sodass
\begin{equation}
  x = Y x_y + Z x_z = H^T \left(\begin{array}{c} x_y \\ x_z \end{array}\right)
\end{equation}
gilt.
Sei $d \in \kr{A} \subseteq \R^n$.
Dann gibt es $d_y \in R^m$ und $d_z \in \R^{n-m}$ mit $d = Y d_y + Z d_z$.
Es gilt
\begin{equation}
Ad = AY d_y + AZ d_z = AH^T \left(\begin{array}{c} d_y \\ d_z \end{array}\right)
   = (R^T \ 0) \left(\begin{array}{c} d_y \\ d_z \end{array}\right)
   = R^T d_y
\end{equation}
und somit wegen der Invertierbarkeit von der Matrix $R$
\begin{equation}
  Ad = 0 \quad \Leftrightarrow \quad d_y = 0
  \quad \Leftrightarrow \quad d = Z d_z,
\end{equation}
d.\,h. $\kr{A} = \im{Z}$.
Damit ist diese Matrix $Z$ die Nullraum-Matrix von $A$.

Wie in~\eqref{prob:opt_prob_unrestr_mithilfe_nullraum_matrix}
k�nnen wir unser Problem~\eqref{prob:quad_opt_prob_mit_lin_gl_nebenbed}
so umformulieren:
\begin{equation}
\min_{z \in \R^{n-m}} f(w + Zz) =: F(z),
\end{equation}
wobei $w$ eine spezielle L�sung der Gleichung $Ax = b$ ist.

\begin{align}
  F(z) & = \frac{1}{2}(w+Zz)^T Q (w+Zz) + q^T(w+(Zz)) \\
       & = \frac{1}{2}w^T Qw + \frac{1}{2}w^T Q Zz
         + \frac{1}{2}z^T ZQw + \frac{1}{2}z^T ZQZz + q^T w+ q^T Zz
\end{align}

