\chapter{SQP-Verfahren}

\section{Einf�hrung}

\chapter{Einf�hrung}
%\addchap*{Einf�hrung} %use this to remove the number for this chapter.

Die nichtlineare Optimierung ist ein bedeutendes Gebiet der Mathematik.
Sie findet immer wieder Anwendungen in den schwierigen Problemen
der Technik und der Wirtschaft.
Viele Verfahren wurden entwickelt,
um nichtlineare Optimierungsprobleme zu l�sen.
In dieser Arbeit werden zwei Verfahren betrachtet und verglichen,
n�mlich das halbglatte Newton-Verfahren und das SQP-Verfahren.

Das SQP-Verfahren geh�rt zu den bekanntesten Verfahren
der nichtlinearen Optimierung.
In seiner Doktorarbeit im Jahr 1963 hat Wilson
das erste SQP-Verfahren entwickelt.
Es wird bis heute in vielen Optimierungsproblemen als
Standardwerkzeug angewendet und wurde inzwischen sehr viel
weiterentwickelt.

Das halbglatte Newton-Verfahren ist nicht so bekannt wie das SQP-Verfahren.
Es basiert aber auf das bekannte Newton-Verfahren.
�ber dieses Verfahren hat Ulbrich in \cite{ulbrich} geschrieben:
``It was not predictable in 2000 that ten years later semismooth Newton
methods would be one of the most important approaches for solving inequality
constrained optimization problems in function spaces.''
% Das war nicht vorhersehbar im Jahr 2000, dass 10 Jahre sp�ter das halbglatte
% Newton-Verfahren einer der wichtigsten Ans�tze geworden ist, um
% Optimierungsprobleme mit Ungleichungsnebenbedingungen
% in Funktionenr�umen zu l�sen



Wir haben gesehen, dass das SQP-Verfahren
quadratische Optimierungsprobleme mit linearen Restriktionen
als Teilprobleme hat.
Wir werden jetzt aufkl�ren, wie wir diese Probleme l�sen werden.

\begin{definition}
\emph{(Quadratische Optimierungsprobleme mit linearen Restriktionen)}
\begin{align}
  \min_{x \in \R^n}\ &
    \frac{1}{2} \langle Qx,x \rangle + \langle q,x \rangle \\
  \nb & Ax = b \notag \\
      & Gx \leq r \notag
\end{align}
\end{definition}




%\begin{figure}[htbp]
%\centering
%\includegraphics[width=0.7\textwidth]{TUBerlin_Logo} % Datei in "bilder/"
%\caption{Titel der Abbildung} 
%\label{fig:bild1}
%\end{figure}

%In der Abbildung~\ref{fig:bild1}\footnote{vgl. Zitat A\cite{ItoKun}} %
% referenzA muss im Verzeichnis literatur in der Datei bib.bib enthalten sein
%ist zu sehen, dass ...
