\chapter{Das halbglatte Newton-Verfahren}

\section{Einf�hrung}

Hier f�ge ich mal eine Tabelle ein
\begin{table}[htbp]
\centering
\begin{tabular}{l|l|l|l}
SpalteA & SpalteB & SpalteC & SpalteD \\
\midrule
InhaltA1 & InhaltB1 & InhaltC1 & InhaltD1 \\
InhaltA2 & InhaltB2 & InhaltC2 & InhaltD2 \\
InhaltA3 & InhaltB3 & InhaltC3 & InhaltD3
\end{tabular}
\caption{Beispiel einer Tabelle}
\label{tab:tabelle1}
\end{table}

Wie man in der Tabelle~\ref{tab:tabelle1} sehen kann ...

\newpage

\section{Active-Set-Strategie}

Problem:	
\begin{align}
  \min & \left( f(x) + \frac{\lambda}{2} |x|^2 \right) \\
       & a \leq x \leq b
\end{align}

Verfahren: Active-Set-Strategie

\begin{enumerate}
  \item $k := 0, x^k := x^0$
  \item Bestimme
        \begin{align}
          \A_+^k & := \{ i | x_i^k > b_i \} \\
          \A_-^k & := \{ i | x_i^k < a_i \} \\
          \I^k   & := \{ i | a_i \leq x_i^k \leq b_i \}
        \end{align}
  \item L�se das Problem
        \begin{align}
          \nabla f(x^{k+1}) + \lambda x^{k+1} + \mu^{k+1} & = 0 \\
          x_i^{k+1} & = b_i \quad \text{ f�r } i \in \A_+^k \\
          x_i^{k+1} & = a_i \quad \text{ f�r } i \in \A_-^k \\
          \mu^{k+1} & = 0   \quad \text{ f�r } i \in \I^k
        \end{align}
  \item Falls
        \begin{align}
          0 & = \nabla f(x^{k+1}) + \lambda x^{k+1} + \mu^{k+1} \\
          \mu^{k+1} & = \max(0,\mu^{k+1}+\lambda(x^{k+1}-b)) +
                         \min(0,\mu^{k+1}+\lambda(x^{k+1}-a))
        \end{align}
        $\Rightarrow$ STOP
  \item $x^k := x^{k+1}, k := k+1$ $\Rightarrow$ 2. %TODO use reference, not 2.
\end{enumerate}
