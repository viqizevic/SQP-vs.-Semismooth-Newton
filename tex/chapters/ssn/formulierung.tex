
Gegeben seien $x_1, x_2 \in \R$.
Es gilt
\begin{equation}
  \begin{array}{c}
    x_1, x_2 \geq 0 \\
    x_1 x_2 = 0
  \end{array}
  \qquad \Leftrightarrow \qquad
  \min(x_1,x_2) = 0
\end{equation}

Wenn wir also die Bedingungen
\begin{equation}
  \mu \geq 0, \quad Gx \leq r \quad \text{und} \quad \mu^T(Gx-r) = 0
\end{equation}
haben, k�nnen wir sie als
\begin{equation}
  \min( \mu, r-Gx ) = 0
\end{equation}
schreiben.

Das Aktive-Menge-Verfahren kann
als das halbglatte Newton-Verfahren interpretiert werden.
