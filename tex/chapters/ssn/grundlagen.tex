Wir betrachten erst mal die allgemeine Definition
einer halbglatten\footnote{englisch: semismooth} Funktion.

\begin{definition}
Seien $X, Z$ reelle Banachr�ume und sei $D \subset X$ offen.
Eine Funktion $F : D \subset X \rightarrow Z$ hei�t
Newton-differenzierbar in Punkt $x \in D$,
falls es eine Umgebung $U(x) \subset D$ von $x$
und eine Abbildung $G : U(x) \rightarrow L(X,Z)$ gibt,
so dass
\begin{equation}
  \lim_{|h| \rightarrow 0} \frac{|F(x+h)-F(x)-G(x+h)h|_Z}{|h|_X} = 0
\end{equation}
Die Familie
\begin{equation}
  \{ G(u)\,|\, u \in U(x) \}
\end{equation}
hei�t N-Ableitung von $F$ in $x$.
\end{definition}

\begin{definition}
Seien $X, Z$ reelle Banachr�ume und sei $D \subset X$ offen.
Eine Funktion $F : D \subset X \rightarrow Z$ hei�t
halbglatt in $x \in D$,
falls $F$ Newton-differenzierbar in $x$ ist und
\begin{equation}
  \lim_{t \rightarrow 0^+} G(x+th)h
\end{equation}
einheitlich in $|h| = 1$ existiert.
\end{definition}

Nun kommt die Definition in endlichdimensionalen R�umen.

\begin{definition}
Sei $F : \R^m \rightarrow \R^n$ lokal Lipschitz-stetig.
Sei $D_F$ die Menge aller Punkte, wo $F$ differenzierbar ist.
F�r $x \in \R^m$ definieren wir
\begin{equation}
  \partial_B F(x) := \left\{ J\  |\ 
    J = \lim_{x_i \rightarrow x, x_i \in D_F} \nabla F(x_i) \right\}
\end{equation}
und die allgemeine Ableitung in $x$
\begin{equation}
  \partial F(x) := \co \partial_B F(x),
\end{equation}
wobei $\co$ f�r die konvexe H�lle steht.
\end{definition}

\begin{definition}
$F : \R^m \rightarrow \R^n$ hei�t halbglatt in $x \in \R^m$,
falls $F$ lokal Lipschitz-stetig in $x$ ist und
\begin{equation}
  \lim_{V \in \partial F(x+th'), h' \rightarrow h, t \rightarrow 0^+} Vh'
\end{equation}
f�r alle $h \in \R^m$ existiert.
\end{definition}

\begin{theorem}
Sei $\xopt$ eine L�sung des Problems $F(x) = 0$
und $F$ sei Newton-differenzierbar in $\xopt$ mit N-Ableitung $G$.
Falls $G$ nichtsingul�r f�r alle $x \in U(\xopt)$ ist
und $\{ \|G(x)^{-1}\|\, | \, x \in U(\xopt) \}$ beschr�nkt ist,
dann konvergiert die Newton-Iteration
\begin{equation}
  x^{k+1} := x^k - G(x^k)^{-1} F(x^k)
\end{equation}
superlinear gegen $\xopt$
unter der Bedingung, dass $|x^0 - \xopt|$ gen�gend klein ist.
\end{theorem}

\begin{algorithm}
\emph{(Das halbglatte Newton-Verfahren)}
\begin{enumerate}
  \item W�hle einen Startpunkt $x^0$ und ein Abbruchkriterium $\epsilon > 0$.
        Setze $k := 0$.
  \item Ist $\|\nabla f(x^k)\| < \epsilon$ \label{list:stop_criteria_ssn}
        $\Rightarrow$ STOP.
  \item W�hle $V_k \in \partial \nabla f (x^k)$
  \item Berechne die L�sung $d$ des linearen Gleichungssystems
        \begin{equation}
          V_k d = - \nabla f(x^k).
        \end{equation}
        Setze $d^k := d$.
  \item Setze $x^{k+1} := x^k + d^k$ und $k := k+1$ $\Rightarrow$
        Gehe zu Schritt~\ref{list:stop_criteria_ssn}.
\end{enumerate}
\end{algorithm}

Das halbglatte Newton-Verfahren hat also das gleiche Ziel
wie bei dem Newton-Verfahren,
n�mlich ein nichtlineares Gleichungssystem zu l�sen. 
Der Unterschied liegt darin, dass hier die Funktion nicht
unbedingt so glatt ist.