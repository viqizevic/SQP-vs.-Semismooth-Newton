\begin{testproblem}
(Steuerungsproblem, vgl. Beispiel~6.2.6 in \cite[S.~216~f.]{alt})\\
Es ist ein Produktionsplanungsproblem mit beschr�nktem Lager zu betrachten.
In einem Zeitraum $[0,T]$ mit $T > 0$
soll eine Produktion so gesteuert werden,
dass die Gesamtkosten minimal sind.
Seien
\begin{itemize}
  \item $z(t)$ die Anzahl der zur Zeit $t$ am Lager vorhandenen Produkte,
  \item $r(t)$ die Nachfrage nach den Produkten zur Zeit $t$,
  \item $u(t)$ die Produktionsrate zur Zeit $t$.
\end{itemize}
Die �nderung der verf�gbaren Lagermenge $z(t)$
sei durch die Lagerbilanzgleichung
\begin{equation}
  \dot{z}(t) = u(t) - r(t), \quad \forall t \in [0,T]
\end{equation}
mit dem Anfangslagerbestand $z(0) = a \geq 0$ gegeben.
Die Produktions- und Lagehaltungskosten zur Zeit $t$ seien
$\frac{1}{2} \rho_u u(t)^2 + \rho_z z(t)$
mit Konstanten $\rho_u, \rho_z > 0$.
\end{testproblem}
