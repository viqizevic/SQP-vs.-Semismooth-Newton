% Die Titelseite
% Im folgenden kommen ein paar Variablen, die auszuf�llen sind
% Bisher steht dort nur Musterinhalt
% Au�erdem m�ssen zei Dateien erstellt werden, Bild/Logo/Emblem des Fachgebietes
% sowie der Universit�t

\newcommand{\trtitle}{Vergleich zwischen halgblattem Newton-Verfahren und
SQP-Verfahren}
\newcommand{\trtype}{Bachelorarbeit}
\newcommand{\trauthor}{Vicky H. Tanzil}
\newcommand{\trstrasse}{Lehrter Stra�e 68}
\newcommand{\trmatrikelnummer}{308789}
\newcommand{\trort}{10557 Berlin}
\newcommand{\trbetreuer}{Prof. Dr. Fredi Tr�ltzsch}
%\newcommand{\trprof}{Prof. Dr. Fredi Tr�ltzsch}
\newcommand{\trfachgebiet}{Optimierung bei partiellen Differentialgleichungen}
%\newcommand{\trinstitut}{Mathematik}
\newcommand{\trfakultaet}{II Mathematik und Naturwissenschaften}
\newcommand{\truni}{Technische Universit�t Berlin}
\newcommand{\trdate}{\today}

\thispagestyle{empty}

% Kopfzeile mit Logos.
% Eventuell die \hspace{} je nach Logogr��e anpassen
%\begin{tabular}{lcr}
\begin{tabular}{l}
  \includegraphics[scale=0.8,width=2cm]{TUBerlin_Logo} %& %
  % dein_unilogo.jpg/.eps im Verzeichnis "bilder" ablegen
  %\hspace{2cm} \truni \hspace{2cm} &
  %\includegraphics[scale=0.8]{dein_fglogo} % dein_fglogo.jpg/.eps im
  % Verzeichnis "bilder" ablegen, Fachgebietslogo
  \\
\end{tabular}

\rule{\textwidth}{0.4pt}

\vspace{2.5cm}
\begin{center}
  \textbf{\LARGE \trtitle}
\end{center}
\vspace{2cm}

\begin{center}
  \textbf{\trtype} \\
  am Fachgebiet \trfachgebiet \\
%  \trprof \\
  Institut f�r \trinstitut \\
  Fakult�t \trfakultaet \\
  \truni \\[0.5cm]
  vorgelegt von \\
  \textbf{\trauthor}
\end{center}

\vspace{1cm}


\begin{center}
\begin{tabular}{ll}
Betreuer: & \trbetreuer \\
\end{tabular}
\end{center}

\vfill

\begin{tabular}{l}
\trauthor \\
Matrikelnummer:  \trmatrikelnummer \\
\trstrasse \\
\trort
\end{tabular}

\rule{\textwidth}{0.4pt}
